\documentclass[a4paper,11pt]{article}
\usepackage{fullpage,color,xcolor,natbib}
\color{black}
\author{Li Ying}
\title{How to Research}
\bibliographystyle{unsrt}

\begin{document}
\maketitle


\subsection*{How to read a paper}
{
Three-Pass Paper Reading Method \cite{readpaper}:
\begin{description}
\item[Pass 1] - Get General Idea [5-10 mins]  \hfill \\
Reading steps: 
\begin{enumerate}
\item Carefully read tile, abstraction and introduction
\item Read the section and sub-section headings, but ignore everything else
\item Glance at the mathematical content (if any) to determine the underlying theoretical foundations
\item Read the conclusions
\item Glance over the references, mentally ticking off the ones you've already read
\end{enumerate}
After reading questions:
\begin{enumerate}
\item Category: measurement; analysis; presentation; prototype description;
\item Context: other related papers; theorical basis;
\item Correctness: valid assumption?
\item Contributions: contribution?
\item Clarity: well written?
\end{enumerate}
\item[Pass 2] - Grasp Contents [30-50 mins] \hfill \\
Reading steps:
\begin{enumerate}
\item Look carefully at the figures, diagrams and other illustrations
\item Mark relevant unread references for further reading
\end{enumerate}
After reading: be able to summarize the thrust and present to other with supporting evidence
\item[Pass 3] - Understand Details [1-3 hours]\hfill \\
Reading: try to re-implement the paper! \\
After reading: be able to re-construct the paper from memory!
\begin{enumerate}
\item implicit assumptions
\item missing citations
\item potential issues with experimental or analytical techniques
\end{enumerate}
\end{description}	
}


\bibliography{h2r}
\end{document}

