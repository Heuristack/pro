\documentclass[a4paper,11pt]{article}
\usepackage{fullpage,color,xcolor,natbib, mathtools}
\color{black}
\author{Li Ying}
\title{Data-Parallel Computing on Commodity Clusters \\ \large{A Literature Survey on the MapReduce Trend} }
\bibliographystyle{unsrt}

\begin{document}
\maketitle
\tableofcontents

\section{Programming Models and Execution Engines}

\subsection*{MapReduce: Simplified Data Processing on Large Clusters}
{\color{cyan} {\color{magenta} Cited by: 10636}

{\color{black} MapReduce\cite{mapred}} is a 
programming model and an 
associated implementation 
for processing and generating large data sets. 

Users specify a 
{\color{red} \em map} function that 
processes a key/value pair to generate a set of intermediate key/value pairs, and a 
{\color{red} \em reduce} function that 
merges all intermediate values associated with the same intermediate key. 

Many real world tasks are expressible in this model, as shown in the paper.

Programs written in this functional style are 
automatically parallelized and executed 
on a large cluster of commodity machine.

The {\color{red} \bf run-time system} 
takes care of the details of 
{\color{red} \em partitioning the input data}, 
{\color{red} \em scheduling the program's execution across a set of machines}, 
{\color{red} \em handling machine failures}, and 
{\color{red} \em managing the required inter-machine communication}. 

This allows programmers without any experience with parallel and distributed systems to 
easily utilize the resources of a large distributed system.

Our implementation of MapReduce runs on a large cluster of commodity machines and is highly scalable: 
a typical MapReduce computation processes many terabytes of data on thousands of machines. 

Programmers find the system easy to use: 
hundreds of MapReduce programs have been {\color{red} \em implemented} and 
upwards of one thousand MapReduce jobs are {\color{red} \em executed} on Google's clusters every day.

}

\subsection*{Dryad: distributed data-parallel programs from sequential building blocks}
{\color{cyan} {\color{magenta} Cited by: 1420}

{\color{black} Dryad\cite{dryad}} is a 
{\color{red} \em general-purpose distributed execution engine}
for coarse-grain data-parallel applications. 

A Dryad application combines 
computational `vertices' with 
communication `channels' 
to form a dataflow graph. 

Dryad 
runs the application by 
executing the vertices of this graph on a set of available computers, 
communicating as appropriate through files, TCP pipes, and shared-memory FIFOs.

The vertices provided by the application developer are 
quite simple and are usually written as 
sequential programs with no thread creation or locking. 

Concurrency 
arises from Dryad scheduling vertices to run simultaneously 
on multiple computers, or on multiple CPU cores within a computer.

The application can 
discover the size and placement of data at run time, and 
modify the graph as the computation progresses to make efficient use of the available resources.

Dryad is designed to 
scale 
from powerful multi-core single computers, 
through small clusters of computers, 
to data centers with thousands of computers. 

The Dryad execution engine 
handles all the difficult problems of 
creating a large distributed, concurrent application: 
scheduling the use of computers and their CPUs, 
recovering from communication or computer failures, and 
transporting data between vertices.

}

\subsection*{FlumeJava: Easy, Efficient Data-Parallel Pipelines}
{\color{cyan} {\color{magenta} Cited by: 143}

MapReduce and similar systems significantly ease the task of writing data-parallel code. 

However, 
many real-world computations require {\color{red} \em a pipeline of MapReduces}, and 
programming and managing such pipelines can be difficult. 

We present 
{\color{black} FlumeJava\cite{flumejava}}, 
a Java library that makes it easy to develop, test, and run efficient data-parallel pipelines. 

At the core of the FlumeJava library are 
a couple of classes that represent 
{\color{red} \em immutable parallel collections}, 
each supporting a modest number of {\color{red} \em operations} 
for processing them in parallel. 

Parallel collections and their operations present a 
simple, high-level, uniform abstraction 
over different data representations and execution strategies. 

To enable parallel operations to run efficiently,
FlumeJava defers their evaluation, 
instead internally constructing an 
{\color{red} \em execution plan dataflow graph}. 

When the final results of the parallel operations are eventually needed, 
FlumeJava 
first {\color{red} \em optimizes the execution plan}, and 
then {\color{red} \em executes the optimized operations} on appropriate underlying primitives (e.g., MapReduces). 

The combination of 
{\color{red} \em high-level abstractions for parallel data and computation,
deferred evaluation and optimization, and 
efficient parallel primitives}
yields an easy-to-use system that approaches the efficiency of hand-optimized pipelines. 

FlumeJava is in active use by hundreds of pipeline developers within Google.

}


\subsection*{Distributed computing in practice: The Condor experience}
{\color{cyan} {\color{magenta} Cited by: 1422}

Since 1984, 
the Condor project has enabled ordinary users to do extraordinary computing. 

Today, 
the project continues to explore the social and technical problems of 
cooperative computing on scales ranging from the desktop to the world-wide computational grid. 

In this chapter, we 
provide the history and philosophy of the {\color{black} Condor project\cite{condor}} and 
describe how it has interacted with other projects and evolved along with the field of distributed computing. 

We 
outline the core components of the Condor system and 
describe how the technology of computing must correspond to social structures.

Throughout, we 
reflect on the lessons of experience and 
chart the course traveled by research ideas as they grow into production systems.

}

\subsection*{Map-Reduce-Merge: Simplified Relational Data Processing on Large Clusters}
{\color{cyan} {\color{magenta} Cited by: 575}

Map-Reduce is a programming model 
that enables easy development of scalable parallel applications 
to process vast amounts of data on large clusters of commodity machines.

Through a simple interface with two functions, map and reduce,
this model facilitates parallel implementation of many real-world tasks such as 
data processing for search engines and machine learning.

However, 
this model does not directly support processing multiple related heterogeneous datasets. 

While processing relational data is a common need, this limitation causes difficulties
and/or inefficiency when Map-Reduce is applied on relational operations like joins.

We improve 
Map-Reduce into a new model called 
{\color{black} Map-Reduce-Merge\cite{mapredmerge}}. 

It adds to Map-Reduce a Merge phase that can efficiently merge data already partitioned and sorted (or
hashed) by map and reduce modules. 

We also demonstrate that 
this new model can express relational algebra operators
as well as implement several join algorithms.

}


\subsection*{CIEL: a universal execution engine for distributed data-flow computing}
{\color{cyan} {\color{magenta} Cited by: 118}

This paper introduces 
{\color{black} CIEL\cite{ciel}}, 
a universal execution engine
for distributed data-flow programs. 

Like previous execution engines, 
CIEL 
{\color{red} \em masks the complexity of distributed programming}. 

Unlike those systems, a 
CIEL job can make data-dependent control-flow decisions, which 
enables it to compute iterative and recursive algorithms.

We have also developed Skywriting, a 
Turing complete scripting language that runs directly on CIEL.

The execution engine provides 
transparent fault tolerance and 
distribution to Skywriting scripts and 
high-performance code written in other programming languages.

We have 
deployed CIEL on a cloud computing platform, and 
demonstrate that it achieves scalable performance 
for both iterative and non-iterative algorithms.	

}

\subsection*{Spark: Cluster Computing withWorking Sets}
{\color{cyan} {\color{magenta} Cited by: 302}

MapReduce and its variants 
have been highly successful 
in implementing large-scale data-intensive applications on commodity clusters. 

However, 
most of these systems are built around an acyclic data flow model 
that is not suitable for other popular applications. 

This paper 
focuses on one such class of applications: 
those that reuse a working set of data across multiple parallel operations.

This includes many 
iterative machine learning algorithms, as well as 
interactive data analysis tools. 

We propose 
a new framework called 
{\color{black} Spark\cite{spark}} 
that supports these applications
while retaining the scalability and fault tolerance of MapReduce. 

To achieve these goals, 
Spark introduces an abstraction called resilient distributed datasets (RDDs).

An RDD is a 
read-only collection of objects 
partitioned across a set of machines 
that can be rebuilt if a partition is lost. 

Spark 
can outperform Hadoop by 10x in iterative machine learning jobs, and c
an be used to interactively query a 39 GB dataset with sub-second response time.

}

\subsection*{Resilient Distributed Datasets: A Fault-Tolerant Abstraction for In-Memory Cluster Computing}
{\color{cyan} {\color{magenta} Cited by: 242}

We present 
{\color{black} Resilient Distributed Datasets (RDDs)\cite{rdd}}, 
a distributed memory abstraction that 
lets programmers perform in-memory computations on large clusters 
in a fault-tolerant manner. 

RDDs are motivated by two types of applications that 
current computing frameworks handle inefficiently: 
iterative algorithms and interactive data mining tools. 

In both cases, 
keeping data in memory 
can improve performance by an order of magnitude.

To achieve fault tolerance efficiently, 
RDDs provide a restricted form of shared memory, 
based on coarsegrained transformations rather than fine-grained updates to shared state. 

However, 
we show that 
RDDs are expressive enough to capture a wide class of computations, including
recent specialized programming models for iterative jobs, such as Pregel, and 
new applications that these models do not capture.

We have implemented RDDs in a system called Spark, 
which we evaluate through a variety of user applications and benchmarks.

}

\subsection*{Optimizing Shuffle Performance in Spark}
{\color{cyan} {\color{magenta} Not published!}

Spark [6] is a cluster framework that performs in-memory computing, 
with the goal of outperforming disk-based engines like Hadoop [2]. 

As with other distributed data processing platforms, 
it is common to collect data in a many-to-many fashion, 
a stage traditionally known as the shuffle phase. 

In Spark, many sources of inefficiency exist in the shuffle phase that, 
once addressed, potentially promise vast performance improvements. 

In this paper {\color{black} \cite{sparkshuffle}}, 
we identify the bottlenecks in the execution of the current design, and 
propose alternatives that solve the observed problems. 

We evaluate our results in terms of 
application level throughput.

}

\subsection*{Twister: a runtime for iterative mapreduce}
{\color{cyan} {\color{magenta} Cited by: 413}

MapReduce programming model has simplified the implementation of many data parallel applications. 

The simplicity of the programming model and 
the quality of services provided by many implementations of MapReduce 
attract a lot of enthusiasm among distributed computing communities. 

From the years of experience in applying MapReduce to various scientific applications 
we identified a set of 
extensions to the programming model and 
improvements to its architecture that will 
expand the applicability of MapReduce to more classes of applications. 

In this paper, 
we present 
the programming model and the architecture of 
{\color{black} Twister\cite{twister}} 
an enhanced MapReduce runtime that 
supports iterative MapReduce computations efficiently. 

We also show 
performance comparisons of Twister with other similar runtimes 
such as Hadoop and DryadLINQ 
for large scale data parallel applications.

}

\subsection*{DryadLINQ: A System for General-Purpose Distributed Data-Parallel Computing 
Using a High-Level Language}
{\color{cyan} {\color{magenta} Cited by: 485}

{\color{black} DryadLINQ\cite{dryadlinq}} 
is a system and a set of language extensions that 
enable a new programming model for large scale distributed computing. 

It generalizes previous execution environments such as SQL, MapReduce, and Dryad 
in two ways: 
by adopting an expressive data model of strongly typed .NET objects; and 
by supporting general-purpose imperative and declarative operations on datasets 
within a traditional high-level programming language.

A DryadLINQ program 
is a sequential program composed of 
LINQ expressions performing arbitrary side-effect-free transformations on datasets, and 
can be written and debugged using standard .NET development tools. 

The DryadLINQ system 
automatically and transparently 
translates the data-parallel portions of the program 
into a distributed execution plan which is passed to the Dryad execution platform. 

Dryad, 
which has been in continuous operation for several years on production clusters made up of thousands of computers, 
ensures efficient, reliable execution of this plan.


We describe the 
implementation of the DryadLINQ compiler and runtime. 

We evaluate DryadLINQ on a varied set of programs drawn from 
domains such as web-graph analysis, large-scale log mining, and machine learning. 

We show that 
excellent absolute performance can be attained - 
a general-purpose sort of 1012 Bytes of data executes in 319 seconds on a 240-computer, 960-disk cluster - 
as well as demonstrating near-linear scaling of execution time on representative applications as 
we vary the number of computers used for a job.

}

\subsection*{Shark: SQL and Rich Analytics at Scale}
{\color{cyan} {\color{magenta} Cited by: 58}

{\color{black} Shark\cite{shark}}
is a new data analysis system that 
marries query processing with complex analytics on large clusters. 

It leverages 
a novel distributed memory abstraction 
to provide a unified engine that can run 
SQL queries and sophisticated analytics functions (e.g., iterative machine learning) at scale, 
and efficiently recovers from failures mid-query. 

This allows Shark to run 
SQL queries up to 100$\times$ faster than Apache Hive, and 
machine learning programs more than 100$\times$ faster than Hadoop.

Unlike previous systems, 
Shark shows that 
it is possible to achieve these speedups while retaining a MapReduce-like execution engine, and 
the fine-grained fault tolerance properties that such engine provides. 

It extends such an engine in several ways, 
including 
{\em column-oriented in-memory storage} and 
{\em dynamic mid-query replanning}, 
to effectively execute SQL. 

The result is a system that matches the speedups reported for MPP analytic databases over MapReduce, while 
offering fault tolerance properties and complex analytics capabilities that they lack.

}

\section{Resource Management: Sharing and Isolation}

\subsection*{Mesos: A Platform for Fine-Grained Resource Sharing in the Data Center}
{\color{cyan} {\color{magenta} Cited by: 224}

We present 
Mesos\cite{mesos}, 
a platform for sharing commodity clusters 
between multiple diverse cluster computing frameworks, 
such as Hadoop and MPI. 

Sharing 
improves cluster utilization and 
avoids per-framework data replication.

Mesos shares resources in a fine-grained manner,
allowing frameworks to achieve data locality by taking turns reading data stored on each machine. 

To support the sophisticated schedulers of today's frameworks,
Mesos introduces a distributed two-level scheduling mechanism called resource offers. 

Mesos decides how many resources to offer each framework, 
while frameworks decide which resources to accept and which computations to run on them. 

Our results show that 
Mesos 
can achieve near-optimal data locality when sharing the cluster among diverse frameworks, 
can scale to 50,000 (emulated) nodes, and 
is resilient to failures.

}

\subsection*{Nexus: A common substrate for cluster computing}
{\color{cyan} {\color{magenta} Mesos's original name}

Cluster computing has become mainstream, 
resulting in the rapid creation and adoption of diverse cluster computing frameworks. 

We believe that 
no single framework will be optimal for all applications, and that 
organizations will instead want to run multiple frameworks in the same cluster. 

Furthermore, 
to ease development of new frameworks, 
it is critical to identify common abstractions and modularize their architectures. 

To achieve these goals, 
we propose 
{\color{black} Nexus\cite{nexus}}, 
a low-level substrate that provides 
isolation and efficient resource sharing across frameworks running on the same cluster, 
while giving each framework maximum control over the scheduling and execution of its jobs. 

Nexus fosters innovation in the cloud 
{\em by letting organizations run new frameworks alongside existing ones} and 
{\em by letting framework developers focus on specific applications} 
{\em rather than building one-size-fits-all frameworks}.

}

\subsection*{Apache hadoop yarn: Yet another resource negotiator}
{\color{cyan} {\color{magenta} Cited by: 25}

The initial design of Apache Hadoop was tightly focused on 
running massive, MapReduce jobs 
to process a web crawl. 

For increasingly diverse companies, 
Hadoop has become the data and computational agora - 
the de facto place where data and computational resources are shared and accessed. 

This broad adoption and ubiquitous usage has stretched the initial design well beyond its intended target, 
exposing two key shortcomings: 
1) tight coupling of a specific programming model with the resource management infrastructure, 
forcing developers to abuse the MapReduce programming model, and 
2) centralized handling of jobs' control flow, 
which resulted in endless scalability concerns for the scheduler.

In this paper, 
we summarize the 
design, development, and current state of deployment of 
the next generation of Hadoop's compute platform: 
{\color{black} YARN\cite{yarn}}. 

The new architecture we introduced 
decouples the programming model from the resource management infrastructure, and
delegates many scheduling functions (e.g., task faulttolerance)
to per-application components. 

We provide
experimental evidence demonstrating the improvements we made, 
confirm improved efficiency by reporting the experience of running YARN on production environments
(including 100\% of Yahoo! grids), and 
confirm the flexibility claims 
by discussing the porting of several programming frameworks onto YARN viz. 
Dryad, Giraph, Hoya, Hadoop MapReduce, REEF, Spark, Storm, Tez.

}

\subsection*{Omega: flexible, scalable schedulers for large compute clusters}
{\color{cyan} {\color{magenta} Cited by: 42}

Increasing scale and the need for rapid response to changing requirements 
are hard to meet with current 
{\color{red} \em monolithic cluster scheduler architectures}.

This 
restricts the rate at which new features can be deployed, 
decreases efficiency and utilization, and will eventually 
limit cluster growth.

We present
{\color{black} a novel approach\cite{omega}}
to address these needs using 
parallelism, shared state, and lock-free optimistic concurrency control.

We 
compare this approach to existing cluster scheduler designs, 
evaluate how much interference between schedulers occurs and how much it matters in practice, 
present some techniques to alleviate it, and finally 
discuss a use case highlighting the advantages of our approach – 
all driven by real-life Google production workloads.

}


\subsection*{Scope: easy and efficient parallel processing of massive data sets}
{\color{cyan} {\color{magenta} Cited by: 443}

Companies providing cloud-scale services 
have an increasing need to store and analyze massive data sets 
such as search logs and click streams. 

For cost and performance reasons, 
processing is typically done on 
large clusters of shared-nothing commodity machines. 

It is imperative to 
develop a programming model that 
hides the complexity of the underlying system but provides flexibility by 
allowing users to extend functionality to meet a variety of requirements.

In this paper,  
we present 
a new declarative and extensible scripting language, 
SCOPE (Structured Computations Optimized for Parallel Execution)\cite{cosmos-scope}, 
targeted for this type of massive data analysis.

The language is designed for ease of use with no explicit parallelism,
while being amenable to efficient parallel execution on large clusters. 

SCOPE borrows several features from SQL. 
Data is modeled as sets of rows composed of typed columns. 
The select statement is retained with inner joins, outer joins, and aggregation allowed. 
Users can easily define their own functions and implement their own versions of operators: 
extractors (parsing and constructing rows from a file), 
processors (row-wise processing),
reducers (group-wise processing), and 
combiners (combining rows from two inputs). 

SCOPE supports nesting of expressions
but also allows a computation to be specified as a series of steps,
in a manner often preferred by programmers. 

We also describe
how scripts are compiled into efficient, parallel execution plans and executed on large clusters.

}

\subsection*{Fuxi: a Fault-Tolerant Resource Management and Job Scheduling System at Internet Scale}
{\color{cyan} {\color{magenta}}

Scalability and fault-tolerance 
are two fundamental challenges for all 
distributed computing at Internet scale. 

Despite many recent advances from both academia and industry, 
these two problems are still far from settled. 

In this paper, we present 
{\color{black} Fuxi\cite{fuxi}}, 
a resource management and job scheduling system that is capable of handling the kind of workload 
at Alibaba where hundreds of terabytes of data are generated and analyzed everyday 
to help optimize the company's business operations and user experiences. 

We employ several novel techniques to enable Fuxi to perform efficient scheduling of hundreds of thousands of concur-
rent tasks over large clusters with thousands of nodes: 

1) an incremental resource management protocol that 
supports multi-dimensional resource allocation and data locality; 
2) user-transparent failure recovery where 
failures of any Fuxi components will not impact the execution of user jobs; and
3) an effective detection mechanism and a multi-level black-listing scheme that 
prevents them from affecting job execution. 

Our evaluation results demonstrate that 95\% and 91\% scheduled CPU/memory utilization can be fulfilled under
synthetic workloads, and Fuxi is capable of achieving 2.36T-B/minute throughput in GraySort. 

Additionally,  the same Fuxi job only experiences approximately 16\% slowdown under a 5\% fault-injection rate. 
The slowdown only grows to 20\% when we double the fault-injection rate to 10\%. 
Fuxi has been deployed in our production environment since 2009, and 
it now manages hundreds of thousands of server nodes.

}


\section{Scheduling: Locality and Fairness}

\subsection*{Quincy: fair scheduling for distributed computing clusters}
{\color{cyan} {\color{magenta} Cited by: 293}

This paper addresses the problem of 
scheduling concurrent jobs on clusters where application data is stored on the computing nodes.

This setting, in which 
scheduling computations close to their data is crucial for performance, 
is increasingly common and arises in systems such as 
MapReduce, Hadoop, and Dryad as well as many grid-computing environments. 

We argue that 
data-intensive computation 
benefits from a fine-grain resource sharing model that 
differs from the coarser semi-static resource allocations 
implemented by most existing cluster computing architectures. 

The problem of
scheduling with locality and fairness constraints has 
not previously been extensively studied 
under this resource-sharing model.

We introduce 
a powerful and flexible new framework 
for scheduling concurrent distributed jobs with fine-grain resource sharing.

The scheduling problem is mapped to a 
graph datastructure, where 
edge weights and capacities encode the competing demands of 
data locality, fairness, and starvation-freedom, and 
a standard solver computes the optimal online schedule according to a global cost model. 

We evaluate 
our implementation of this framework, which we call 
{\color{black} Quincy\cite{quincy}}, 
on a cluster of a few hundred computers using a varied workload of data- and CPU-intensive jobs. 

We 
evaluate Quincy against an existing queue-based algorithm and 
implement several policies for each scheduler, with and without fairness constraints.

Quincy gets better fairness when fairness is requested,
while substantially improving data locality. 

The volume of data transferred across the cluster is reduced by up to a factor of 3.9 
in our experiments, 
leading to a throughput increase of up to 40\%.

}

\subsection*{Delay scheduling: a simple technique for achieving locality and fairness in cluster scheduling}
{\color{cyan} {\color{magenta} Cited by: 429}

As organizations start to use data-intensive cluster computing systems 
like Hadoop and Dryad 
for more applications,
there is a growing need to share clusters between users.

However, 
there is a conflict between 
fairness in scheduling and 
data locality (placing tasks on nodes that contain their input data). 

We illustrate this problem 
through our experience designing a fair scheduler for a 600-node Hadoop cluster at Facebook. 

To address the conflict between locality and fairness, 
we propose 
a simple algorithm called 
{\color{black} delay scheduling\cite{delay}}: 
when the job that should be scheduled next according to fairness cannot launch a local task, 
it waits for a small amount of time, letting other jobs launch tasks instead.

We find that 
delay scheduling 
achieves nearly optimal data locality in a variety of workloads and can 
increase throughput by up to 2x while preserving fairness. 

In addition, the
simplicity of delay scheduling makes it 
applicable under a wide variety of scheduling policies beyond fair sharing.	

}

\subsection*{Improving MapReduce Performance in Heterogeneous Environments}
{\color{cyan} {\color{magenta} Cited by: 789}

MapReduce is emerging as an important programming model 
for large-scale data-parallel applications such as
web indexing, data mining, and scientific simulation.

Hadoop is an open-source implementation of MapReduce enjoying wide adoption and is 
often used for short jobs where low response time is critical. 

Hadoop's performance is closely tied to its task scheduler, 
which implicitly assumes that 
cluster nodes are homogeneous and tasks make progress linearly, and 
uses these assumptions to decide when to speculatively re-execute tasks that appear to be stragglers. 

In practice, 
the homogeneity assumptions do not always hold. 

An especially compelling setting where this occurs is a virtualized data center, 
such as Amazon's Elastic Compute Cloud (EC2).

We show that 
{\em Hadoop's scheduler can cause 
severe performance degradation in heterogeneous environments}. 

We design
a new scheduling algorithm, 
{\color{black} Longest Approximate Time to End (LATE)\cite{late}}, 
that is highly robust to heterogeneity.

LATE can improve Hadoop response times 
by a factor of 2 
in clusters of 200 virtual machines on EC2.	

}

\subsection*{Sparrow: Distributed, Low Latency Scheduling}
{\color{cyan} {\color{magenta} Cited by: 12}

Large-scale data analytics frameworks are 
shifting towards shorter task durations and larger degrees of parallelism to provide low latency. 

Scheduling highly parallel jobs that 
complete in hundreds of milliseconds poses a major 
challenge for task schedulers, 
which will need to
schedule millions of tasks per second on appropriatemachines
while offering millisecond-level latency and high availability. 

We demonstrate that a 
decentralized, randomized sampling approach provides near-optimal performance
while avoiding the throughput and availability limitations of a centralized design. 

We implement and deploy our scheduler, 
{\color{black} Sparrow\cite{sparrow}}, 
on a 110-machine cluster and 
demonstrate that Sparrow performs within 12\% of an ideal scheduler.

}

\subsection*{Assigning Tasks for Efficiency in Hadoop}
{\color{cyan} {\color{magenta} Cited by: 37}

In recent years Google's MapReduce has emerged as a leading large-scale data processing architecture. 

Adopted by companies such as Amazon, Facebook, Google, IBM and Yahoo! in daily use, and 
more recently put in use by several universities, 
it allows parallel processing of huge volumes of data over cluster of machines. 

Hadoop is a free Java implementation of MapReduce. 
In Hadoop,files are split into blocks and replicated and spread over all servers in a network.
Each job is also split into many small pieces called tasks. 
Several tasks are processed on a single server, and a job is not completed until all the assigned tasks are finished.
A crucial factor that affects the completion time of a job is the particular assignment of tasks to servers. 

Given a placement of the input data over servers, 
one wishes to find the assignment that minimizes the completion time. 

In this paper, 
{\color{black} an idealized Hadoop model \cite{hadoopmodel}}
is proposed 
to investigate the Hadoop task assignment problem. 

It is shown that 
{\em there is no feasible algorithm to find the optimal Hadoop task assignment} 
unless P = NP. 

Assignments that are computed by the 
{\em round robin algorithm} 
inspired by the current Hadoop scheduler are shown to 
{\em deviate from optimum} 
by a multiplicative factor in the worst case. 

A fow-based algorithm is presented that 
computes assignments that are optimal to within an additive constant.	

}

\subsection*{BAR: An Efficient Data Locality Driven Task Scheduling Algorithm for Cloud Computing}
{\color{cyan} {\color{magenta} Cited by: 34}

Large scale data processing is increasingly common in cloud computing systems 
like MapReduce, Hadoop, and Dryad 
in recent years. 

In these systems, 
files are split into many small blocks and 
all blocks are replicated over several servers. 

To process files efficiently, 
each job is divided into many tasks and 
each task is allocated to a server to deals with a file block. 

Because network bandwidth is a scarce resource in these systems, 
{\em enhancing task data locality (placing tasks on servers that contain their input blocks)}
is crucial 
for the job completion time. 

Although there have been many approaches on improving data locality, 
most of them 
either are greedy and ignore global optimization, 
or suffer from high computation complexity. 

To address these problems, 
we propose 
{\color{black} a heuristic task scheduling algorithm called BAlance-Reduce(BAR) \cite{bar}}, 
in which 
an initial task allocation will be produced at first, then 
the job completion time can be reduced gradually by tuning the initial task allocation. 

By taking a global view, 
BAR can adjust data locality dynamically 
according to network state and cluster workload. 

The simulation results show that BAR 
is able to 
deal with large problem instances in a few seconds and 
outperforms previous related algorithms in term of the job completion time.	

}

\subsection*{Leveraging Endpoint Flexibility in Data-Intensive Clusters}
{\color{cyan} {\color{magenta} Cited by: 7}

Many applications do not constrain the destinations of their network transfers. 

New opportunities emerge when such transfers contribute a large amount of network bytes. 

By choosing the endpoints to avoid congested links, 
completion times of these transfers as well as that of others without similar flexibility can be improved. 

In this paper, we focus on 
leveraging the flexibility in 
replica placement during writes to cluster file systems (CFSes), 
which account for almost 
{\em half of all cross-rack traffic} 
in data-intensive clusters. 

The replicas of a CFS write can be placed in any subset of machines 
as long as they are in multiple fault domains and ensure a balanced use of storage throughout the cluster.

We study CFS interactions with the cluster network, analyze optimizations for replica placement, and propose 
{\color{black} Sinbad\cite{sinbad}}, 
a system that identifies imbalance and adapts replica destinations to navigate around congested links. 

Experiments on EC2 and trace-driven simulations show that 
{\em block writes} 
complete 1.3 $\times$ (respectively, 1.58 $\times$) faster 
as the network becomes more balanced. 

As a col-lateral benefit, end-to-end completion times of data-intensive jobs improve as well. 

Sinbad does so with little impact on the long-term storage balance.	

}

\subsection*{Paragon: QoS-Aware Scheduling for Heterogeneous Datacenters}
{\color{cyan} {\color{magenta} Cited by: 37}

Large-scale datacenters (DCs) host tens of thousands of diverse applications each day. 

However, interference between colocated workloads and the difficulty to match applications to one of the many
hardware platforms available can degrade performance, violating the
quality of service (QoS) guarantees that many cloud workloads require.

While previous work has identified the impact of heterogeneity and interference, 
existing solutions are computationally intensive, 
cannot be applied online and do not scale beyond few applications.

We present 
{\color{black} Paragon\cite{paragon}}, 
an online and scalable DC scheduler that is heterogeneity and interference-aware. 

Paragon is derived from robust analytical methods and instead of profiling each application in detail, 
it leverages information the system already has about applications it has previously seen. 

It uses 
collaborative filtering techniques 
to quickly and accurately classify an unknown, incoming workload
with respect to heterogeneity and interference in multiple shared resources, 
by identifying similarities to previously scheduled applications.

The classification allows Paragon to greedily schedule applications 
in a manner that minimizes interference and maximizes server utilization. 

Paragon scales to tens of thousands of servers with marginal scheduling overheads in terms of time or state.

We evaluate Paragon with a wide range of workload scenarios, 
on both small and large-scale systems, including 1,000 servers on EC2.

For a 2,500-workload scenario, 
Paragon enforces performance guarantees for 91\% of applications, while significantly improving utilization.

In comparison, 
heterogeneity-oblivious, interference-oblivious and least-loaded schedulers only provide similar guarantees for 14\%, 11\% and 3\% of workloads. 

The differences are more striking in oversubscribed scenarios where resource efficiency is more critical.

}

\section{Coordination: Lock Service, Group Messaging and Leader Election}

\subsection*{The Chubby lock service for loosely-coupled distributed systems}
{\color{cyan} {\color{magenta} Cited by: 546}

We describe our experiences with the 
{\color{black} Chubby lock service\cite{chubby}}, 
which is intended to provide 
coarse-grained locking as well as reliable (though low-volume) 
storage for a loosely-coupled distributed system. 

Chubby provides an interface much like a 
{\em distributed file system with advisory locks}, 
but the design emphasis is on 
{\em availability and reliability}, 
as opposed to 
{\em high performance}. 

Many instances of the service have been used for over a year,
with several of them each handling a few tens of thousands of clients concurrently. 

The paper 
describes the initial design and expected use, 
compares it with actual use, and 
explains how the design had to be modified to accommodate the differences.	

}

\subsection*{ZooKeeper: Wait-free coordination for Internet-scale systems}
{\color{cyan} {\color{magenta} Cited by: 370}

In this paper, 
we describe 
{\color{black} ZooKeeper\cite{zookeeper}}, 
a service for coordinating processes of distributed applications. 

Since ZooKeeper is part of critical infrastructure, 
ZooKeeper aims to provide 
a simple and high performance kernel 
for building more complex coordination primitives at the client. 

It incorporates elements from 
group messaging, shared registers, and distributed lock services 
in a replicated, centralized service. 

The interface exposed by ZooKeeper has the wait-free aspects of shared registers with
an event-driven mechanism similar to cache invalidations of distributed file systems 
to provide a simple, yet powerful coordination service.

The ZooKeeper interface enables a high-performance service implementation. 

In addition to the wait-free property, 
ZooKeeper provides a per client guarantee of FIFO execution of requests and linearizability for all requests
that change the ZooKeeper state. 

These design decisions 
enable the implementation of 
a high performance processing pipeline with read requests being satisfied by local servers. 

We show for the target workloads, 2:1 to 100:1 read to write ratio, that 
ZooKeeper can handle tens to hundreds of thousands of transactions per second.

This performance allows ZooKeeper to be used extensively by client applications.	

}

\section{Communication Management: Link, Flow, Transfer and Network Scheduling}

\subsection*{Efficient Coflow Scheduling with Varys}
{\color{cyan} {\color{magenta} Cited by: 0}

Communication in 
data-parallel applications often involves a collection of 
parallel flows. 

Traditional techniques to 
optimize flow-level metrics 
do not perform well in optimizing such collections,
because the network is largely agnostic to application-level requirements.

The recently proposed coflow abstraction 
bridges this gap and 
creates new opportunities for 
{\color{red} \em network scheduling}. 

In this paper,
we address 
{\color{red} \em inter-coflow scheduling}
for two different objectives:
{\color{red} \em decreasing communication time of data-intensive jobs} and
{\color{red} \em guaranteeing predictable communication time}. 

We introduce the
{\color{red} concurrent open shop scheduling with coupled resources} 
problem,
analyze its complexity, and 
propose effective heuristics to optimize either objective. 

We present 
{\color{black} Varys \cite{varys}}, 
a system that 
enables data-intensive frameworks to use coflows and the proposed algorithms
while 
{\color{red} maintaining high network utilization} and 
{\color{red} guaranteeing starvation freedom}. 

EC2 deployments and trace-driven simulations
show that 
communication stages complete up to 3.16$\times$ faster on average and 
up to 2$\times$ more coflows meet their deadlines 
using Varys in comparison to per-flow mechanisms. 

Moreover, 
Varys outperforms non-preemptive coflow schedulers by more than 5$\times$.	

}

\subsection*{Decentralized Task-Aware Scheduling for Data Center Networks}
{\color{cyan} {\color{magenta} Cited by: 0}

Most data center applications perform rich and complex tasks 
(e.g., executing a search query or generating a user's wall). 

From a network perspective, these tasks typically comprise multiple flows, which traverse different
parts of the network at potentially different times. 

Existing network resource allocation schemes, 
however, 
treat all these flows in isolation – rather than as part of a task – and therefore only optimize flow-level metrics.

In this paper, 
we show that task-aware network scheduling, which groups flows of a task and schedules them together, 
can reduce both the average as well as tail completion time for typical data center applications.

Based on the network footprint of real applications, 
we motivate the use of a scheduling policy that dynamically adapts the level of multiplexing in the network. 

To apply task-aware scheduling to online applications with small (sub-second) tasks, 
we design and implement 
{\color{black} Baraat\cite{baraat}}, 
a decentralized task-aware scheduling system. 

Through experiments with Memcached on a small testbed and large-scale simulations, 
we show that, 
compared to existing schemes, 
Baraat can reduce tail task completion times by 
60\% for data analytics workloads and 
40\% for search workloads.

}

\subsection*{Coflow: A networking abstraction for cluster applications}
{\color{cyan} {\color{magenta} Cited by: 7}

Cluster computing applications – 
frameworks like MapReduce and 
user-facing applications like search platforms –
have application-level requirements and higher-level abstractions to express them. 

However, 
there exists 
{\color{red} \em no networking abstraction}
that can take advantage of the rich semantics readily available from these data parallel applications.

We propose 
{\color{black} coflow\cite{coflow}}, 
a networking abstraction to express the communication requirements of prevalent data parallel
programming paradigms. 

Coflows make it easier for the applications to 
{\color{red} \em convey their communication semantics to the network}, 
which in turn enables the network to better optimize common communication patterns.

}

\subsection*{Managing data transfers in computer clusters with orchestra}
{\color{cyan} {\color{magenta} Cited by: 109; year: 2011;}

Cluster computing applications like MapReduce and Dryad 
{\color{red} transfer massive amounts of data between their computation stages}. 

These transfers can have a significant impact on job performance, 
accounting for 
{\color{red} more than 50\% of job completion times}. 

Despite this impact, 
there has been relatively little work on 
optimizing the performance of these data transfers, 
with networking researchers traditionally focusing on per-flow traffic management. 

We address this limitation by 
proposing a {\color{black} global management architecture and a set of algorithms\cite{orchestra}} that 
(1) 
{\color{red} improve the transfer times of common communication patterns}, 
such as broadcast and shuffle, and 
(2) 
{\color{red} allow scheduling policies at the transfer level}, 
such as prioritizing a transfer over other transfers. 

Using a prototype implementation, 
we show that 
our solution 
improves broadcast completion times by up to {\color{red} $4.5 \times$} 
compared to the status quo in Hadoop. 

We also show that
transfer-level scheduling can 
reduce the completion time of high priority transfers by {\color{red} $1.7 \times$}.

}
{\color{blue} \small 


How does the system work in brief:
When an application wishes to perform a transfer in Orchestra,
it invokes an API that launches a TC for that transfer. 
{\color{red} \em The TC registers with the ITC to obtain its share of the network.}
The ITC periodically consults a scheduling policy (e.g., FIFO, priority) to
assign shares to the active transfers, and sends these to the TCs.
Each TC can divide its share among its source-destination pairs as it wishes 
(e.g., to choose a distribution graph for a broadcast).
The ITC also updates the transfers' shares periodically as the set of active transfers changes. 
Finally, each TC unregisters itself when its transfer ends. 
Note that we assume a cooperative environment, where all the jobs use TCs and obey the ITC.	

}
{\color{blue} \small
\begin{itemize}
\item trend: distributed computing framework on large clusters for big data processing
\item ignorance: resource scheduling and managing focused on compute and storage but no network
\item intuition: managing and optimizing network activity is critical for improving job performance
\item problem: 
\item argument: transfer but not flow
\item solution/proposal: global control architecture to manage intra- and inter- transfer activities
\end{itemize}	
}

\subsection*{An Architecture for Internet Data Transfer}
{\color{cyan} {\color{magenta} Cited by: 112}

This paper 
presents the 
design and implementation of
{\color{black} DOT\cite{dot}}, 
a flexible architecture for data transfer. 

This architecture
separates content negotiation from the data transfer itself. 

Applications determine what data they need to send and then use a new transfer service to send it. 

This transfer service 
acts as a common interface 
between applications and the lower-level network layers, 
facilitating innovation both above and below. 

The transfer service 
frees developers from re-inventing transfer mechanisms in each new application. 

New transfer mechanisms, in turn, can be easily deployed without modifying existing applications.

We discuss the 
benefits that arise from separating data transfer into a service and 
the challenges this service must overcome. 

The paper then examines the implementation of
DOT and its plugin framework for creating new data transfer mechanisms. 

A set of 
microbenchmarks 
shows that the DOT prototype performs well, and that 
the overhead it imposes is unnoticeable in the wide-area. 

End-to-end experiments 
using more complex configurations 
demonstrate DOT's ability to implement effective, new data delivery
mechanisms underneath existing services. 

Finally, we
evaluate a production mail server modified to use DOT 
using trace data gathered from a live email server. 

Converting the mail server required only 184 lines-of-code changes to the server, and 
the resulting system reduces the bandwidth needed to send email by up to 20\%.	

}

\subsection*{ShuffleWatcher: Shuffle-aware scheduling in multi-tenant MapReduce clusters}
{\color{cyan} {\color{magenta}}
	
MapReduce clusters are usually multi-tenant (i.e., shared among multiple users and jobs) 
for improving cost and utilization. 

The performance of jobs in a multitenant MapReduce cluster is greatly impacted by the 
all-Map-to-all-Reduce communication, or Shuffle, 
which saturates the cluster’s hard-to-scale network bisection bandwidth. 

Previous schedulers optimize Map input locality but do not consider the Shuffle, 
which is often the dominant source of traffic in MapReduce clusters.

We propose 
ShuffleWatcher\cite{shufflewatcher}, 
a new multi-tenant MapReduce scheduler that shapes and reduces Shuffle
traffic to improve cluster performance (throughput and job turn-around times), 
while operating within specified fairness constraints. 

ShuffleWatcher employs three key techniques. 
First, 
it curbs intra-job Map-Shuffle concurrency to shape Shuffle traffic 
by delaying or elongating a job's Shuffle based on the network load. 
Second, 
it exploits the reduced intra-job concurrency and the flexibility engendered 
by the replication of Map input data for fault tolerance 
to preferentially assign a job's Map tasks to localize the Map output to as few nodes as possible.
Third, 
it exploits localized Map output and delayed Shuffle 
to reduce the Shuffle traffic by preferentially assigning a job's Reduce tasks to the nodes containing its Map output. 

ShuffleWatcher leverages opportunities that are unique to multi-tenancy, such
overlapping Map with Shuffle across jobs rather than within a job, and trading-off intra-job concurrency for
reduced Shuffle traffic. 

On a 100-node Amazon EC2 cluster running Hadoop, 
ShuffleWatcher improves cluster throughput 
by 39-46\% and job turn-around times by 27-32\% 
over three state-of-the-art schedulers.	

}

\subsection*{MegaPipe: A New Programming Interface for Scalable Network I/O}
{\color{cyan} {\color{magenta}}

We present 
MegaPipe\cite{megapipe}, 
a new API for efficient, scalable network I/O for message-oriented workloads. 

The design of MegaPipe centers around the abstraction of a channel –
a per-core, bidirectional pipe between the kernel and user space, 
used to exchange both I/O requests and event notifications.

On top of the channel abstraction, we introduce
three key concepts of MegaPipe: 
{\color{red} \em partitioning, lightweight socket (lwsocket), and batching.}

We implement MegaPipe in Linux and adapt memcached and nginx. 

Our results show that, by embracing a clean-slate design approach, 
MegaPipe is able to exploit new opportunities for improved performance and ease of programmability. 

In microbenchmarks on an 8-core server with 64 B messages, 
MegaPipe outperforms baseline Linux between 29\% (for long connections) and 582\% (for short connections). 

MegaPipe improves the performance of a modified version of memcached between 15\% and 320\%. 

For a workload based on real-world HTTP traces, 
MegaPipe boosts the throughput of nginx by 75\%.

}


\section{Stream Processing and Real-time Analytics}

\subsection*{MapReduce Online}
{\color{cyan} {\color{magenta} Cited by: 447}

MapReduce is a popular framework for data-intensive distributed computing of batch jobs. 

To simplify fault tolerance, 
many implementations of MapReduce 
materialize the entire output of each map and reduce task
before it can be consumed. 

In this paper, we propose a
{\color{black} modified MapReduce architecture\cite{mapredonline}} 
that allows data to be pipelined between operators. 

This extends the MapReduce programming model beyond batch processing, and
can reduce completion times and improve system utilization for batch jobs as well. 

We present 
a modified version of the Hadoop MapReduce framework that supports online aggregation, which 
allows users to see ``early returns'' from a job as it is being computed. 

Our Hadoop Online Prototype (HOP) also supports continuous queries, which
enable MapReduce programs to be written for applications
such as event monitoring and stream processing.

HOP retains the fault tolerance properties of Hadoop and can 
run unmodified user-defined MapReduce programs.	

}

\subsection*{Discretized Streams: An Efficient and Fault-Tolerant Model for Stream Processing on Large Clusters}
{\color{cyan} {\color{magenta} Cited by: 47}

Many important ``big data'' applications need to process data arriving in real time. 

However, 
current programming models for distributed stream processing are relatively low-level, 
often leaving the user to worry about consistency of state across the system and fault recovery.

Furthermore, 
the models that provide fault recovery do so in an expensive manner, 
requiring either hot replication or long recovery times. 

We propose 
a new programming model, 
{\color{black} discretized streams (D-Streams)\cite{sparkstream}}, that
offers a high-level functional programming API, 
strong consistency, and efficient fault recovery. 

D-Streams support 
a new recovery mechanism that improves efficiency over the traditional replication and 
upstream backup solutions in streaming databases: parallel recovery of lost state across the cluster. 

We have prototyped D-Streams in an extension to the Spark cluster computing framework called Spark Streaming, 
which lets users seamlessly intermix streaming, batch and interactive queries.

}

\subsection*{Storm at twitter}
{\color{cyan} {\color{magenta}}

This paper describes the use of Storm at Twitter. 

{\color{black} Storm\cite{storm}} 
is a realtime fault-tolerant and distributed stream data processing system.

Storm is currently being used to run various critical computations in Twitter at scale, and in real-time. 

This paper describes the architecture of Storm and its methods for distributed scale-out and fault-tolerance. 

This paper also describes how queries (aka. topologies) are executed in Storm, 
and presents some operational stories based on running Storm at Twitter. 

We also present results from an empirical evaluation demonstrating the resilience of
Storm in dealing with machine failures. 

Storm is under active development at Twitter and we also present some potential directions for future work.

}

\subsection*{MillWheel: Fault-Tolerant Stream Processing at Internet Scale}
{\color{cyan} {\color{magenta} Cited by: 15}

{\color{black} MillWheel\cite{millwheel}}
is a framework for building low-latency data-processing applications that is widely used at Google. 

Users specify a directed computation graph and application code for individual nodes, and
the system manages persistent state and the continuous flow of records, 
all within the envelope of the framework's fault-tolerance guarantees.

This paper describes MillWheel's programming model as well as its implementation. 

The case study of a continuous anomaly detector in use at Google serves 
to motivate how many of MillWheel's features are used. 

MillWheel's programming model provides a notion of logical time, 
making it simple to write time-based aggregations.

MillWheel was designed from the outset with fault tolerance and scalability in mind. 

In practice, we find that 
MillWheel's unique combination of scalability, fault tolerance, and a versatile programming model 
lends itself to a wide variety of problems at Google.

}


\section{Storage Systems: Distributed File Systems and Parallel Databases}

\subsection*{The Google File System}
{{\color{cyan} {\color{magenta} Cited by: 4227}

We have designed and implemented the 
{\color{black} Google File System\cite{gfs}},
a scalable distributed file system for large distributed data-intensive applications. 

It provides fault tolerance while running on inexpensive commodity hardware, and 
it delivers high aggregate performance to a large number of clients.

While sharing many of the same goals as previous distributed file systems, 
our design has been driven by observations of our application workloads and technological environment,
both current and anticipated, that reflect a marked departure from some earlier file system assumptions. 

This has led us to reexamine traditional choices and explore radically different design points.
The file system has successfully met our storage needs.
It is widely deployed within Google as the storage platform for the generation and processing of data used by our service
as well as research and development efforts that require large data sets. 

The largest cluster to date provides 
{\color{red} \em hundreds of terabytes of storage across thousands of disks on over a thousand machines}, and 
it is concurrently accessed by hundreds of clients.

In this paper, we 
present file system interface extensions designed to support distributed applications, 
discuss many aspects of our design, and 
report measurements from both micro-benchmarks and real world use.

}

\subsection*{The Hadoop Distributed File System}
{{\color{cyan} {\color{magenta} Cited by: 828}

The 
{\color{black} Hadoop Distributed File System (HDFS)\cite{hdfs}} 
is designed 
to store very large data sets reliably, and 
to stream those data sets at high bandwidth to user applications. 

In a large cluster, 
thousands of servers both host directly attached storage and execute user application tasks. 

By distributing storage and computation across many servers, 
the resource can grow with demand while remaining economical at every size. 

We 
describe the architecture of HDFS and 
report on experience using HDFS to manage {\color{red} 25 petabytes} of enterprise data at Yahoo!.

}

\subsection*{Bigtable: A distributed storage system for structured data}
{\color{cyan} {\color{magenta} Cited by: 2822}

{\color{black} Bigtable\cite{bigtable}}
is a distributed storage system 
for managing structured data that is designed to scale to a very large size: 
petabytes of data across thousands of commodity servers. 

Many projects at Google store data in Bigtable, 
including web indexing, Google Earth, and Google Finance. 

These applications place very different demands on Bigtable, 
both in terms of 
data size (from URLs to web pages to satellite imagery) and 
latency requirements (from backend bulk processing to realtime data serving). 

Despite these varied demands, 
Bigtable has successfully provided a flexible,
high-performance solution for all of these Google products. 

In this article, 
we describe 
the simple data model provided by Bigtable, which gives clients dynamic control over data layout and format,
and we describe 
the design and implementation of Bigtable.

}

\subsection*{Spanner: Google's globally distributed database}
{\color{cyan} {\color{magenta} Cited by: 230}

{\color{black} Spanner\cite{spanner}}
is Google's 
{\color{red} \em scalable, multiversion, globally distributed, and synchronously replicated}
database. 

It is the first system to 
distribute data at global scale and 
support externally-consistent distributed transactions.

This article 
describes how Spanner is structured, 
its feature set, 
the rationale underlying various design decisions, and 
a novel time API that exposes clock uncertainty. 

This API and its implementation are 
critical to supporting external consistency and a variety of powerful features: 
{\color{red} \em nonblocking reads in the past, lockfree snapshot transactions, and atomic schema changes}, 
across all of Spanner.

}

\subsection*{Parallel database systems: the future of high performance database systems}
{\color{cyan} {\color{magenta} Cited by: 1340}

Parallel database machine architectures have evolved 
from the use of exotic hardware 
to a software parallel dataflow architecture 
based on conventional shared-nothing hardware. 

These new designs 
provide impressive 
{\color{red} \em speedup and scaleup}
when processing relational database queries. 

This paper \cite{pdb}
reviews the techniques used by such systems, and 
surveys current commercial and research systems	

}

\section{Intra-Node Acceleration: SMP and HSA}

\subsection*{Evaluating mapreduce for multi-core and multiprocessor systems}
{\color{cyan} {\color{magenta} Cited by: 640}

This paper 
evaluates the suitability of the MapReduce model 
for multi-core and multi-processor systems. 

MapReduce was created by Google 
for application development
on data-centers with thousands of servers. 

It allows programmers to write functional-style code that is 
automatically parallelized and scheduled in a distributed system.

We describe 
{\color{black} Phoenix\cite{phoenix07}}, 
an implementation of MapReduce 
for shared-memory systems that 
includes a programming API and an efficient runtime system. 

The Phoenix runtime automatically manages 
{\em thread creation,  dynamic task scheduling, data partitioning, and fault tolerance}
across processor nodes. 

We 
study Phoenix with multi-core and symmetric multiprocessor systems and 
evaluate its performance potential and error recovery features. 

We also compare MapReduce code to code written in lower-level APIs such as P-threads. 

Overall, we establish that, given a careful implementation, 
MapReduce is a promising model for scalable performance on shared-memory systems 
with simple parallel code.

}

\subsection*{Phoenix rebirth: Scalable MapReduce on a large-scale shared-memory system}
{\color{cyan} {\color{magenta} Cited by: 138}

Dynamic runtimes can simplify parallel programming 
by automatically managing concurrency and locality without further burdening the programmer. 

Nevertheless, 
implementing such runtime systems for large-scale, shared-memory systems can be challenging. 

This work 
{\color{black} optimizes Phoenix\cite{phoenix09}}, 
a MapReduce runtime for shared-memory multi-cores and multiprocessors, 
on a quad-chip, 32-core, 256-thread UltraSPARC T2+ system with NUMA characteristics. 

We show 
how a multi-layered approach that comprises optimizations 
on the algorithm, implementation, and OS interaction 
leads to significant speedup improvements 
with 256 threads (average of 2.5× higher speedup, maximum of 19×).

We also identify the roadblocks that limit the scalability of parallel runtimes on shared-memory systems, 
which are inherently tied to the OS scalability on large-scale systems.

}

\subsection*{Optimizing MapReduce for Multicore Architectures}
{\color{cyan} {\color{magenta} Cited by: 48}

MapReduce is a programming model for data-parallel programs originally intended for data centers. 

MapReduce simplifies parallel programming, hiding synchronization and task management. 

These properties make it a promising programming model for future processors with many cores, and 
existing MapReduce libraries such as Phoenix have demonstrated that applications written
with MapReduce perform competitively with those written with Pthreads [11].

This paper 
explores the design of the MapReduce data structures for grouping intermediate key/value pairs,
which is often a performance bottleneck on multicore processors. 

The paper finds the 
{\em best choice depends on workload characteristics}, 
such as 
{\em the number of keys used by the application}, 
{\em the degree of repetition of keys}, 
etc. 

This paper also introduces a new MapReduce library, 
{\color{black} Metis\cite{metis}}, 
with a compromise data structure designed to perform well for most workloads. 

Experiments with the Phoenix benchmarks on a 16-core AMD-based server
show that 
Metis' data structure performs better than simpler alternatives, 
including Phoenix.	

}

\subsection*{Map-Reduce for Machine Learning on Multicore}
{\color{cyan} {\color{magenta} Cited by: 715}

We are at the beginning of the multicore era. 

Computers will have increasingly many cores (processors), 
but there is still 
no good programming framework for these architectures, and thus 
no simple and unified way for machine learning to take advantage of the potential speed up. 

In this paper, we develop 
{\color{black} a broadly applicable parallel programming method \cite{mapredlearning}}, 
one that is easily applied to many different learning algorithms. 

Our work is in distinct contrast to the tradition in machine learning of 
designing (often ingenious) ways to speed up a single algorithm at a time. 

Specifically, 
we show that algorithms that fit the Statistical Query model [15] can be written in a certain ``summation form'', 
which allows them to be easily parallelized on multicore computers. 

We adapt Google's map-reduce [7] paradigm to demonstrate this parallel speed up technique 
on a variety of learning algorithms including 
{\em
locally weighted linear regression (LWLR), 
k-means, 
logistic regression (LR), 
naive Bayes (NB), 
SVM, ICA, PCA, 
gaussian discriminant analysis (GDA), 
EM, and 
backpropagation (NN)
}. 

Our experimental results show basically 
{\em linear speedup} 
with an increasing number of processors.	

}


\subsection*{Accelerator: using data parallelism to program GPUs for general-purpose uses}
{\color{cyan} {\color{magenta} Cited by: 288}

GPUs are difficult to program for general-purpose uses. 

Programmers can either learn graphics APIs and convert their applications to use graphics pipeline operations 
or they can use stream programming abstractions of GPUs.

We describe 
{\color{black} Accelerator\cite{accelerator}}, 
a system that uses data parallelism to program GPUs for general-purpose uses instead. 

Programmers use 
a conventional imperative programming language and 
a library that provides only high-level data-parallel operations. 

No aspects of GPUs are exposed to programmers. 

The library implementation compiles the data-parallel operations on the fly 
to optimized GPU pixel shader code and API calls.

We describe the compilation techniques used to do this. 

We evaluate the effectiveness of using data parallelism to program GPUs 
by providing results for a set of compute-intensive benchmarks. 

We 
compare the performance of Accelerator versions of the benchmarks 
against hand-written pixel shaders. 

The speeds of the Accelerator versions are typically 
within 50\% of the speeds of hand-written pixel shader code. 

Some benchmarks significantly outperform C versions on a CPU: 
they are up to 18 times faster than C code running on a CPU.

}


\subsection*{Mars: a MapReduce framework on graphics processors}
{\color{cyan} {\color{magenta} Cited by: 437}

We design and implement 
{\color{black} Mars\cite{mars}}, 
a MapReduce framework, on graphics processors (GPUs). 

MapReduce is a distributed programming framework originally proposed by Google 
for the ease of development of web search applications 
on a large number of commodity CPUs. 

Compared with CPUs, 
GPUs have an order of magnitude higher computation power and memory bandwidth, 
but are harder to program 
since their architectures are designed as  a special-purpose co-processor and their 
programming interfaces are typically for graphics applications. 

As the first attempt to harness GPU's power for MapReduce, 
we 
developed Mars on an NVIDIA G80 GPU,
which contains over one hundred processors, and 
evaluated it in comparison with Phoenix, 
the state-of-the-art MapReduce framework on multi-core CPUs. 

Mars hides the programming complexity of the GPU behind the simple and familiar MapReduce interface.
It is up to {\color{red} \em 16 times} faster than its CPU-based counterpart 
for six common web applications on a quad-core machine.	

}

\section{Conclusion and Future Prevision: Datacenter Computing}

\subsection*{Parallel Data Processing with MapReduce: A Survey}
{\color{cyan} {\color{magenta} Cited by: 127}

A prominent parallel data processing tool MapReduce 
is gaining significant momentum 
from both industry and academia
as the volume of data to analyze grows rapidly. 

While MapReduce is used in many areas 
where massive data analysis is required,
there are still debates on its 
{\em performance}, 
{\em efficiency per node}, and 
{\em simple abstraction}. 

This survey \cite{mapredsurvey}
intends to assist 
the database and open source communities 
in understanding various technical aspects of the MapReduce framework.

In this survey, we 
{\em characterize the MapReduce framework} and 
discuss its inherent 
{\em pros and cons}. 

We then introduce its 
{\em optimization strategies} 
reported in the recent literature.

We also discuss the 
open issues and challenges 
raised on 
parallel data analysis with MapReduce.	

}


\subsection*{The Datacenter Needs an Operating System \cite{dcos}}
{\color{cyan} {\color{magenta} Cited by: 12}

Datacenters have become a major computing platform,
powering not only popular Internet services but also a growing number of scientific and enterprise applications.

We argued that, 
as the use of this new platform grows,
datacenters increasingly need an 
{\em operating system-like software stack}
for the same reasons that single computers did: 
{\em resource sharing between applications and users, data sharing, and abstraction}. 

This kind of software stack is already emerging in an ad-hoc manner, 
but 
{\color{red} \em now is the right time for researchers to take a long-term approach to these problems and 
have a lasting impact on the software infrastructure for this new computing platform}.	

}

\subsection*{The Data Center Is The Computer \cite{dccomputer}}
{\color{cyan} {\color{magenta} Cited by: 12}

Internet services 
are already significant forces in 
searching, retail purchases, music downloads, and auctions. 

One vision of 21st century IT is that 
most users will be accessing such services over a descendant of the cell phone 
rather than running shrink-wrapped software on a descendant of the PC.

The beauty of MapReduce is that any programmer can understand it, and 
its power comes from being able to harness thousands of computers behind that simple interface. 

When paired with the distributed Google File System to deliver data, 
programmers can write simple functions that can do amazing things.

{\color{red} \em 
I predict MapReduce will inspire new ways of thinking about 
the design and programming of large distributed systems. 
}

{\color{red} \em
If MapReduce is the first instruction of the ``data center computer'',
I can't wait to see the rest of the instruction set, 
as well as the data center programming language, the data center operating system, the data center storage systems,
and more.
}

}

\nocite{*}
\bibliography{mapred}
\end{document}

